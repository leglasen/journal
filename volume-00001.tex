\documentclass[a4paper]{article}
% \usepackage[francais]{babel}
\usepackage[utf8]{inputenc}
\usepackage[T1]{fontenc}
\usepackage{datetime}
\usepackage{lmodern}
\usepackage{hyphenat}
\usepackage{microtype}
\usepackage{newspaper}
\usepackage{hyperref}
\usepackage{graphicx}
\usepackage{multicol}
\usepackage{picinpar}
\usepackage{newspaper-mod}
\usepackage{lipsum}
\usepackage{epigraph}
\usepackage{blindtext}
\usepackage[footnote,printonlyused]{acronym}
\usepackage{scrextend}
\usepackage{bookmark}
\usepackage[textwidth=18cm,textheight=26cm]{geometry}

\usepackage[
  backend=bibtex,
  sorting=none
]{biblatex}
\addbibresource{bibliographie.bib}

\setlength{\footnotesep}{0cm}
\setlength{\skip\footins}{0cm}
\date{Le vendredi 27 août 2021}
\currentvolume{1}
\currentissue{1}
\SetPaperName{Le Glas:en}
\SetHeaderName{Le Glasen}
\SetPaperLocation{France Libre}
\SetPaperSlogan{Sonnons le glas de la soumission et de l'indignité}
\SetPaperPrice{Hebdomadaire de Réinformation Gratuit}
\thispagestyle{empty}
\pagenumbering{gobble}

\deffootnote[1.0em]{3.2em}{10em}
{\textsuperscript{\thefootnotemark}\,\enskip}

\begin{document}

% \maketitle

\begin{multicols}{3}

  {\textgoth{\fontsize{15}{60}\selectfont\usefont{LYG}{bigygoth}{m}{n}Le Glas:en}}\hfill
  
  \vspace*{0.1in}
  \begin{center}
    \MakeUppercase{\small Le 27 août 2021 - Volume n°1 }\hfill \\
    \vspace*{0.2in}

    \begin{minipage}{5cm}
    \noindent « Toute l'histoire du contrôle sur le peuple se résume à cela:
    isoler les gens les uns des autres, parce que si on peut les maintenir
    isolés assez longtemps, on peut leur faire croire n'importe
    quoi. »  -- Noam Chomsky
    \vspace*{0.2in}
    \end{minipage}
    
    % \vspace*{0.1in}
    \vspace*{-16pt}
    \rule[1pt]{\hsize}{1pt}
  \end{center}

Le ``Glasen'' ou ``coup de cloche'' permettait d'indiquer l'heure aux
marins.  Toutes les 30 minutes la cloche était sonnée. Au bout de 8
coups de cloche, un autre membre de l'équipage se relayait pour la
faire sonner.

Le navire France est en train de sombrer, le capitaine du navire, élu
``démocratiquement'', n'a que faire de ses matelots. Nous, simples
marins, restons sur le navire partant à la dérive. Relayons nous pour
faire sonner le glas, informons nous et partageons ce que nous savons.

Voici donc la naissance du ``Glasen'', un micro-journal imprimé sur
une feuille A4 recto-verso, dont l'objectif est de diffuser
l'actualité sur les crises de notre époque. Plus largement, il a aussi
pour vocation de réinformer et rééduquer la population pour que cette
dernière puisse reprendre en main sa destinée. Né dans la sombre
période de la crise sanitaire du coronavirus, en l'an 2021, ce journal
sera disponible tous les vendredi soir, avant les manifestations qui
se déroulent le week-end.

Merci à vous tous d'avoir indirectement participé à la naissance de ce
journal. Bonne lecture et continuez la lutte.

\headline{Effets Indésirables}
  
La thérapie génique imposée de force par les gouvernements occidentaux
n'a plus à prouver son danger et ses nombreux effets indésirables
subis par la population. Les chiffres le prouvent un peu plus chaque
semaine en augmentant dangereusement vers les plus hauts sommets du
mauvais goût.

En date du 21/08/2021, la base de donnée de la pharma-covigilance
Européenne Eudravigilance\cite{AdrReports} s'occupant de recevoir les
déclarations d'effets indésirables pour les ``vaccins'' Pfizer
(TOZIMERAN), Moderna (CX-024414), Astrazeneca (CHADOX1) et Janssen
(AD26.COV2.S), reportait \textbf{22414} décès, \textbf{552813} effets
indésirables non résolus, \textbf{873473} effets indésirables résolus,
\textbf{25769} effets indésirables avec séquelles, \textbf{483021}
effets secondaires résolus et \textbf{358093} effets secondaires
déclarés ayant un statut inconnu. À la date d'écriture de cet article,
selon le centre de prévention des maladies
européen\cite{VaccineTracker} environ 280 millions d'européens ont
reçu une première dose et environ 247 millions ont reçu les deux
injections.

Nos confrères États-Uniens ne sont pas en reste. Selon les
\textbf{595620} déclarations remontées par le système
VAERS\cite{OpenVaers}, \textbf{13068} décès ont été déclarés ainsi que
\textbf{54142} hostpitalisations, \textbf{72699} urgences,
\textbf{98761} visites médicales, \textbf{5617} chocs anaphylactiques,
\textbf{4681} paralysies de Bell, \textbf{1607} fausses-couches,
\textbf{5882} arrêts cardiaques, \textbf{4861} cas de myocardie,
\textbf{17228} personnes invalides à vie, \textbf{2738} cas de
thromboses, \textbf{25168} allergies sévères et \textbf{7080}
zonas. Selon USA facts\cite{UsaFacts}, environ 202 millions de
personnes ont reçu une première dose, et environ 172 millions de
personnes ont reçu les deux injections.

Des chiffres similaires sont disponibles outre-manche, au Royaume-Uni,
où les rapports sont accessibles via le système Yellow
Card\cite{YellowCardReport}. Environ 46 millions de premières doses
ont été injectées, et environ 40 millions de secondes injections ont
été réalisées. \textbf{544660} effets indésirables ont été déclarés
pour les traitements géniques Pfizer (\textbf{302146} effets déclarés
dont \textbf{506} décès), Astrazeneca (\textbf{229134} effets déclarés
dont \textbf{1056} décès), Moderna (\textbf{14019} effets déclarés
dont \textbf{17} décès) et d'origine inconnue (\textbf{1036} effets
déclarés dont \textbf{28} décès).

\closearticle

\headline{Défense Juridique}

Depuis presque deux longues années maintenant, l'État français est
passé d'une ``démocratie'' a un état de crise où tout est
possible. Cette situation a commencé en mars 2020 avec la pandémie de
coronavirus. Cet état d'urgence a donné naissance a des lois
liberticides. Il est du devoir de chaque citoyen de se protéger en
conséquence.

Suite à la déclaration du 12 juillet 2021 par Mr Macron, un ``passe
sanitaire'' est imposé ainsi qu'une obligation vaccinale pour certains
citoyens français. Une distinction entre les vaccinés et non vaccinés
se met doucement en place, coupant la France en deux. Les français
non-vaccinés n'auront alors plus accès à certains lieux
publiques. Comme autrefois l'Afrique du Sud avait fait un apartheid
racial, la France invente l'apartheid sanitaire.

Les avocats et juristes ne sont pas restés muets face à cette
situation et ont d'ores et déjà trouvé des parades ainsi que des
solutions à cette situation inconfortable pour une grande majorité de
nos concitoyens.

Maître de Araujo-Recchia a mis à disposition de nombreux modèles de
lettres\cite{AraujoRecchia:InjectionObligatoire} pour aider les salariés ou
agents publiques du domaine de la santé à se défendre contre le
``passe sanitaire'' et l'injection obligatoire. Elle rappelle aussi
dans un article daté du 03 août 2021 que ce document ainsi que son
contrôle n'est pas
légal\cite{AraujoRecchia:PassSanitaire}.

Le Courrier des Stratèges partage aussi cette vision et offre à ses
lecteurs plusieurs modèles de lettres pour aider les salariés
pressurés par le ``passe
sanitaire''\cite{CourrierDesStrateges:PassSanitaire}. Le même journal
nous apprend à nous défendre au moyen des
Prud'hommes\cite{CourrierDesStrateges:PrudHommes} avec un exemple
typique d'échanges attendus. Guillaume Zambrano propose aussi un
formulaire sur son site\cite{CourrierDesStrateges:PrudHommes} pour
demander le maintien du salaire en référé.

Un guide juridique pour la liberté vaccinale et contre le pass
sanitaire a aussi été créé par la Ligue Nationale pour la Liberté des
Vaccinations\cite{InfoVaccin:Guide} contenant des modèles de lettres,
des avis juridiques, des rappels à la loi mais aussi des mémos
récapitulatifs réalisés par Maître de Araujo-Recchia.

D'un point de vue de l'actualité, les préfets d'Ile-de-France ont été
sommés de suspendre le ``passe sanitaire'' sous peine de poursuite
judiciaire, action réalisée par Maîtres Yoann Sibille et Tarek
Koraitem\cite{LeParisien:SuspensionPassSanitaire}. De nombreux recours
ont été aussi envoyés au CEDH\cite{NoPass:CEDH}, les plaintes contre le
gouvernement sont toujours disponibles sur le site de Maître Di
Vizio\cite{DiVizio:PlaintesVeranCastex}.

En Europe aussi les avocats s'activent. Maître Nikos Antoniadis a
déposé un recours de plus de 4300 pages\cite{NikosAntoniadis:Recours}
accusant le gouvernement grec, les experts, les hopitaux ainsi que
les médias d'avoir participé à des actions criminelles contre la
population.

\closearticle

\headline{Pfizerleak}

Les contrats signés par les différents États pour l'accès aux vaccins
durant cette crise ne sont pas accessibles. À vrai dire, les seuls
contrats disponibles publiquement en Europe étaient caviardés. Ce
mystère a été partiellement résolu le 27 juillet 2021 quand ces
contrats ont fuité et ont été partagés\cite{EhdenBiber:Pfizerleak}
par Ehden Biber, un expert en sécurité informatique.

Pas de piratage ici, seulement un document mal protégé et facilement
accessible par toute personne un peu curieuse. Le premier contrat
découvert dans la nature était celui que le Brésil avait signé avec
Pfizer. Puis vint ensuite le contrat signé par le gouvernement de
l'Albanie. Un troisième contrat fut découvert peu de jours après. Ce
dernier s'avère être un original signé par le Brésil au moyen un
certificat numérique. Il permet donc de confirmer l'authenticité d'au
moins un document. Une vingtaine de contrat sont disponibles sur le
canal Telegram\cite{EhdenBiber:Files} d'Ehden Biber, espace qui fut
créé suite à une tentative de blocage du compte Twitter de l'auteur.

Les informations contenues dans ces documents sont d'une importance
cruciale pour la suite du combat. Tout d'abord, ces contrats
n'auraient dû être mis à disposition du public qu'après dix longues
années - ou trente ans pour le cas d'Israël. Une clause souligne le
fait que les acheteurs (les gouvernements) n'ont pas le droit de faire
de la publicité pour des produits qui feraient de l'ombre aux
vaccins. Une autre clause souligne la déresponsabilisation complète du
laboratoire en cas d'effets indésirables, mais aussi en cas de non
fonctionnement du produit sur la maladie. Effectivement, Pfizer tient
à préciser que les effets sur le long terme ne sont pas connus; ni son
efficacité sur la maladie. Pfizer en a aussi profité pour déclarer les
clauses de son contrat plus importantes que les lois du pays où il a
été signé.

Quand on sait que l'accès à ces contrats fut extrèmement limité, même
par les députés européens eux-mêmes, il n'est pas étonnant de voir
apparaitre ce type de clause outre-passant les droits des
États. D'ailleurs, que faut-il faire pour pouvoir signer un tel
affront? Peut-être recevoir des pot-de-vins comme l'ancien ministre de
la santé brésilien, déchu de ses fonctions après avoir reçu 1 million
de dollar\cite{Bresil:Corruption:MinistreSante} par des sociétés
pharmaceutiques. Ou bien encore la commissaire européenne de la santé
qui a reçu environ 4 millions d'euros de la part des mêmes
multinationales\cite{Europe:Corruption:CommissionVaccin}. Conflits
d'intérêts ou corruption, le perdant reste le citoyen.
\closearticle

\headline{Nous contacter}
Nous vous recommandons d'utiliser protonmail pour l'envoi de vos
e-mail. Créez vous un compte gratuit et contactez nous à l'adresse
mail suivante: \textbf{\url{leglasen@protonmail.com}}. Retrouvez-nous
aussi sur Telegram sur le canal
\textbf{\url{https://t.me/leglasen}}. Ce journal utilise Latex et ses
sources sont disponibles à l'adresse
\textbf{\url{https://codeberg.org/leglasen}}. 

Le Glasen est sous license \textbf{\url{CC BY-NC-ND 4.0}}, vous pouvez
librement partager ce document.

\closearticle

\printbibliography[heading=none]

\end{multicols}
\end{document}
